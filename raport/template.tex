\documentclass[10pt, a4paper]{article}

\usepackage[utf8]{inputenc}
\usepackage[T1]{fontenc}
\usepackage[polish]{babel}
\usepackage{geometry}
\geometry{
  top=2cm,
  bottom=2cm,
  left=1.5cm,
  right=1.5cm
}

\usepackage{amsmath, amssymb}
\usepackage{graphicx}
\usepackage{fancyhdr}
\usepackage{titlesec}
\usepackage{biblatex} 
\addbibresource{sample.bib} 

\usepackage{lmodern} 
\usepackage{tabularx}
\usepackage{enumitem}
\usepackage{xcolor}
\usepackage{sectsty} 
\allsectionsfont{\sffamily}

\title{
  \textbf{Projekt z Metod Modelowania Matematycznego 2025} \\
  \large Zadanie 10: Implementacja symulatora układu danego za pomocą transmitancji
}

\author{
  Natalia \textsc{Sampławska 197573} \\
  Martyna \textsc{Penkowska 197926}
}

\date{\today}

\pagestyle{fancy}
\fancyhf{}
\lhead{Modelowanie Matematyczne}
\rhead{\thepage}

\begin{document}

\maketitle

\begin{center}
  \begin{tabular}{l r}
    Okres trwania projektu: & Semestr letni roku akademickiego 2025 \\
    Prowadzący projekt: & dr inż. \textsc{Marek Tatara}
  \end{tabular}
\end{center}

\vspace{0.1cm}

%--------------------------------------------------------------------------------------------------------------------------------------

\section{Cel}

Implementacja symulatora układu danego za pomocą transmitancji: 

\[
G(s) = \frac{a_3 s^3 + a_2 s^2 + a_1 s + a_0}{b_4 s^4 + b_3 s^3 + b_2 s^2 + b_1 s + b_0}
\]

gdzie \( a_i \) oraz \( b_i \) to zmienne parametry modelu, umożliwiającego uzyskanie odpowiedzi czasowych układu na pobudzenia sygnałami: sinusoidalnym, prostokątnym,
piłokształtnym, trójkątnym, impulsem prostokątnym, impulsem jednostkowym i skokiem jednosktkowym. Parametry sygnałów można odpowiednio ustawiać w oknie symulacji. Symulator 
umożliwia określenie stabilności układu oraz wykreślanie charakterystyk częstotliwościowych Bodego - amplitudowej i fazowej. 

%----------------------------------------------------------------------------------------------------------------------------------------------

\section{Zaimplementowane funkcje symulatora}

\begin{tabular}{l l}
	1. & Wybór parametrów obiektu\\ 
	2. & Wybór sygnału wejściowego wraz z wszystkimi jego parametrami\\
	3. & Generowanie charakterystyk Bodego\\
	4. & Implementacja różniczkowania\\
	5. & Przedstawienie graficzne sygnału wejściowego i wyjściowego\\
  6. & Określenie stabilności układu \\
\end{tabular}

%------------------------------------------------------------------------------------------------------------------------------------------

\section{Opis funkcji}

\begin{enumerate}[label=\alph*.]
  \item \textbf{Wybór parametrów obiektu i sygnału wejściowego.} \par\vspace{0.1cm}
  Z poziomu interfejsu można zmienić wszystkie parametry układu – współczynniki transmitancji oraz wybrany sygnał pobudzający wraz z wartościami, które go charakteryzują 
  (amplitudą, częstotliwością, fazą, szerokością impulsu). 

  \vspace{0.1cm}

  \textcolor{blue}{\texttt{check\_input\_values}} --- funkcja sprawdzająca poprawność wprowadzonych danych. Sprawdzenie czy wszystkie współczynniki licznika i mianownika
  są liczbami, czy licznik lub mianownik nie są zerowe, czy rząd licznika nie jest wyższy niż rząd mianownika oraz czy parametry sygnału wejściowego są poprawne: amplituda, 
  częstotliwość i szerokość impulsu większe od zera, faza z zakresu od $-\pi$ do $\pi$. W przypadku błędnych danych funkcja generuje flagę błędu wyświetlającą się w 
  oknie main view. Gdy dane są poprawne wyświetla się komunikat "Correct parameters". Wywołanie funkcji występuje po naciśnięciu przycisku set parameters oraz przed wykonaniem
  symulacji układu.

  \vspace{0.1cm}

  \textcolor{blue}{\texttt{update\_selected\_signal}}, \textcolor{blue}{\texttt{update\_signal\_param\_visibility}} --- funkcje aktualizujące dane sygnału wejściowego. 
  Wybór pobudzenia jest zrealizowany za pomocą radiobuttonów. Po naciśnięciu odpowiedniego przycisku wyświetlają się pod nim parametry danego sygnału, których wartości
  można wprowadzać z klawiatury

  \vspace{0.2cm}
  
  \item \textbf{Okno programu.} \par\vspace{0.1cm}
  Interfejs graficzny jest zrealizowany za pomocą biblioteki PyQt5. Po uruchomieniu kodu otwiera się okno startowe \textcolor{blue}{\texttt{start\_menu}}, w którym
  znajduje się opis założeń projektu. Z niego po naciśnięciu przycisku menu przechodzimy do głównej części programu \textcolor{blue}{\texttt{menu\_display}}. 
  Z tego poziomu zmieniamy parametry transmitancji i pobudzenia. Naciśnięcie przycisku simulate umożliwia przejście do okna symulacji \textcolor{blue}{\texttt{simulation}}, 
  w którym wyświetlają się charakterystyki Bodego, ustawiona postać transmitancji oraz wykresy sygnałów wejściowych i wyjściowych.

  \vspace{0.2cm}
  
  \item \textbf{Pobudzenia wejściowe układu.} \par\vspace{0.1cm}
  Układ może być pobudzany następującymi sygnałami: 
  \begin{itemize}
  \item \textbf{Sinusoidalnym:} \( y(t) = A \sin(2\pi f t + \varphi) \)
  \item \textbf{Prostokątnym:} \( y(t) = A \cdot \operatorname{sgn}\left( \sin(2\pi f t + \varphi) \right) \)
  \item \textbf{Piłokształtnym:} \( y(t) = \left( \frac{2A}{T} \right) (t \bmod T) - A \)
  \item \textbf{Trójkątnym:} \(y(t) = A \left(1 - 4 \left| \frac{t \bmod T}{T} - \frac{1}{2} \right| \right)\)
  \item \textbf{Impulsem prostokątnym:} \( y(t) = 
  \begin{cases}
  A, & \text{dla } 0 < t < pulse width \\
  0, & \text{w przeciwnym razie}
  \end{cases}
  \)
  \item \textbf{Skokiem jednostkowym:} \( y(t) = A \cdot u(t) \)
  \item \textbf{Impulsem jednostkowym:} \(y(t) = A \cdot \delta(t)\)
\end{itemize}

\vspace{0.2cm}
  
  \item \textbf{Wyznaczenie wyjścia układu.} \par\vspace{0.1cm}
  Do wyznaczenia wyjścia wykorzystano różniczkowanie metodą Eulera. 

  \vspace{0.1cm}

  \textcolor{blue}{\texttt{get\_manual\_input\_derivatives}} --- aproksymacja pochodnych sygnału wejściowego 
  \[
\begin{aligned}
\dot{u}[i] &\approx \frac{u[i] - u[i-1]}{\Delta t} \quad &\text{(pierwsza pochodna)} \\
\ddot{u}[i] &\approx \frac{\dot{u}[i] - \dot{u}[i-1]}{\Delta t} \quad &\text{(druga pochodna)} \\
\dddot{u}[i] &\approx \frac{\ddot{u}[i] - \ddot{u}[i-1]}{\Delta t} \quad &\text{(trzecia pochodna)}
\end{aligned}
\]
\noindent gdzie \( \Delta t \) to krok czasowy.

\vspace{0.1cm}

\textcolor{blue}{\texttt{euler\_output}} --- wyznaczenie kolejnych pochodnych sygnału wyjściowego za pomocą metody Eulera

{\footnotesize
\begin{align*}
y^{(4)}[k] &= \frac{a_3 \cdot u^{(3)}[k] + a_2 \cdot u^{(2)}[k] + a_1 \cdot u^{(1)}[k] + a_0 \cdot u[k] - b_3 \cdot y^{(3)}[k-1] - b_2 \cdot y^{(2)}[k-1] - b_1 \cdot y^{(1)}[k-1] - b_0 \cdot y[k-1]}{b_4} \\
y^{(3)}[k] &= y^{(3)}[k-1] + \Delta t \cdot y^{(4)}[k] \\
y^{(2)}[k] &= y^{(2)}[k-1] + \Delta t \cdot y^{(3)}[k] \\
y^{(1)}[k] &= y^{(1)}[k-1] + \Delta t \cdot y^{(2)}[k] \\
y[k] &= y[k-1] + \Delta t \cdot y^{(1)}[k]
\end{align*}
}

Analogicznie wyznaczamy najwyższą pochodną oraz pozostałe pochodne wyjścia dla układów niższych rzędów.

  \vspace{0.2cm}

  \item \textbf{Rysowanie wykresów sygnału wejściowego i wyjściowego.} \par\vspace{0.1cm}
  Do rysowania wykresów wykorzystano bibliotekę matplotlib. Rysowanie wykresów realizują funkcje \textcolor{blue}{\texttt{input\_plot}} i \textcolor{blue}{\texttt{output\_plot}}
  w klasie {\texttt{input\_function}} i {\texttt{output\_compute}}.

  \vspace{0.2cm}
  
  \item \textbf{Wyznaczenie charakterystyk Bodego.} \par\vspace{0.1cm}
    Za narysowanie charakterystyki amplitudowej i fazowej oraz za określenie stabilności
    odpowiedzialna jest klasa {\texttt{bode\_plot}}. Inicjalizacja klasy pobiera potrzebne dane
    oraz przygotowuje zakres kreślonego wykresu. W funkcji \textcolor{blue}{\texttt{plotting\_bode}} obliczne są charakterystyki falowe i amplitudowe
    układu. Dodatkowo, określana jest stabilność na podstawie zapasów fazy i wzmocnienia.
  \vspace{0.2cm}

  \item \textbf{Określenie stabilności układu.} \par\vspace{0.1cm}
  Stabilność układu określana jest w funkcji \textcolor{blue}{\texttt{plotting\_bode}}.
  Układy stabilne charakteryzują się brakiem biegunów dodatnich oraz dodatnimi zapasami fazy i wzmocnienia.
  Zapas fazy oblicza się, kiedy wykres wzmocnienia przechodzi do ujemnych wartości. Natomiast, zapas wzmocnienia określany jest, gdy wykres przecina wartość $-180^{\circ}$.
  Jeżeli taki punkt nie występuje, zapas wzmocnienia przyrównywany jest do $\infty$.
\end{enumerate}

%---------------------------------------------------------------------------------------------------------------------------

\section{Podsumowanie i wnioski}

Założeniem projektu było zaimplementowanie symulatora układu danego transmitancją. Funkcje realizujące różniczkowanie zostały 
zaimplementowane przez autorów projektu (różniczkowanie metodą Eulera). Generowanie sygnałów wejściowych zaimplementowano bez wykorzystania
gotowych funkcji analitycznych z wyjątkiem np.sin wykorzystanej dla sinudoidy i sygnału prostokątnego. Program umożliwia zadawanie 
parametrów symulacyjnych i zmianę parametrów w stosunku do symulacji pokazanej przed chwilą z poziomu interfejsu użytkownika. 

\vspace{0.5cm}

1. Metoda Eulera jest metodą dość prostą w implementacji i bardzo elastyczną w modyfikowaniu dla dowolnego rzędu układu. 
Działanie jest prawidłowe pod warunkiem dobrania odpowiedniego kroku czasowego --- w naszej symulacji dobrany został krok 0.01s. 

\vspace{0.5cm}

2. Łatwość modyfikacji parametrów transmitancji i sygnałów pobudzających umożliwia szybkie testowanie wielu wariantów układu. W tym układów
o różnych rzędach --- od zerowego do czwartego. 

\vspace{0.5cm}

3. Coś o metodzie RK4 jeśli jednak zostanie dołączona z output tests (porównanie etc)

\vspace{0.5cm}

Projekt spełnił wszystkie założone wymagania. Symulator jest dobrym narzędziem do testowania odpowiedzi różnych układów do 4 rzędu 
i można dość łatwo rozbudowywać go dodając inne pobudzenia, czy zwiększając rząd układu. 


\end{document}